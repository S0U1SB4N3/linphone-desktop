\section{Basic buddy status notification}
\label{group__buddy__tutorials}\index{Basic buddy status notification@{Basic buddy status notification}}


This program is a {\itshape very} simple usage example of liblinphone, demonstrating how to initiate S\+IP subscriptions and receive notifications from a sip uri identity passed from the command line.  


This program is a {\itshape very} simple usage example of liblinphone, demonstrating how to initiate S\+IP subscriptions and receive notifications from a sip uri identity passed from the command line. 

~\newline
Argument must be like sip\+:{\tt jehan@sip.\+linphone.\+org} . ~\newline
 ex budy\+\_\+list sip\+:{\tt jehan@sip.\+linphone.\+org} ~\newline
Subscription is cleared on S\+I\+G\+I\+NT ~\newline
 
\begin{DoxyCodeInclude}

\textcolor{comment}{/*}
\textcolor{comment}{buddy\_status}
\textcolor{comment}{Copyright (C) 2010  Belledonne Communications SARL}
\textcolor{comment}{}
\textcolor{comment}{This program is free software; you can redistribute it and/or}
\textcolor{comment}{modify it under the terms of the GNU General Public License}
\textcolor{comment}{as published by the Free Software Foundation; either version 2}
\textcolor{comment}{of the License, or (at your option) any later version.}
\textcolor{comment}{}
\textcolor{comment}{This program is distributed in the hope that it will be useful,}
\textcolor{comment}{but WITHOUT ANY WARRANTY; without even the implied warranty of}
\textcolor{comment}{MERCHANTABILITY or FITNESS FOR A PARTICULAR PURPOSE.  See the}
\textcolor{comment}{GNU General Public License for more details.}
\textcolor{comment}{}
\textcolor{comment}{You should have received a copy of the GNU General Public License}
\textcolor{comment}{along with this program; if not, write to the Free Software}
\textcolor{comment}{Foundation, Inc., 51 Franklin Street, Fifth Floor, Boston, MA  02110-1301, USA.}
\textcolor{comment}{*/}

\textcolor{preprocessor}{#include "linphone/core.h"}

\textcolor{preprocessor}{#include <signal.h>}

\textcolor{keyword}{static} bool\_t running=TRUE;

\textcolor{keyword}{static} \textcolor{keywordtype}{void} stop(\textcolor{keywordtype}{int} signum)\{
        running=FALSE;
\}

\textcolor{keyword}{static} \textcolor{keywordtype}{void} notify\_presence\_recv\_updated (LinphoneCore *lc,  LinphoneFriend *\textcolor{keyword}{friend}) \{
        \textcolor{keyword}{const} LinphoneAddress* friend\_address = linphone_friend_get_address(\textcolor{keyword}{friend});
        \textcolor{keywordflow}{if} (friend\_address != NULL) \{
                \textcolor{keyword}{const} LinphonePresenceModel* model = 
      linphone_friend_get_presence_model(\textcolor{keyword}{friend});
                LinphonePresenceActivity *activity = 
      linphone_presence_model_get_activity(model);
                \textcolor{keywordtype}{char} *activity\_str = linphone_presence_activity_to_string(activity);
                \textcolor{keywordtype}{char} *str = linphone_address_as_string (friend\_address);
                printf(\textcolor{stringliteral}{"New state state [%s] for user id [%s] \(\backslash\)n"}
                                        ,activity\_str
                                        ,str);
                ms\_free(str);
        \}
\}
\textcolor{keyword}{static} \textcolor{keywordtype}{void} new\_subscription\_requested (LinphoneCore *lc,  LinphoneFriend *\textcolor{keyword}{friend}, \textcolor{keyword}{const} \textcolor{keywordtype}{char}* url) \{
        \textcolor{keyword}{const} LinphoneAddress* friend\_address = linphone_friend_get_address(\textcolor{keyword}{friend});

        \textcolor{keywordflow}{if} (friend\_address != NULL) \{
                \textcolor{keywordtype}{char} *str = linphone_address_as_string (friend\_address);
                printf(\textcolor{stringliteral}{" [%s] wants to see your status, accepting\(\backslash\)n"}, str);
                ms\_free(str);
        \}
        linphone_friend_edit(\textcolor{keyword}{friend}); \textcolor{comment}{/* start editing friend */}
        linphone_friend_set_inc_subscribe_policy(\textcolor{keyword}{friend},LinphoneSPAccept); \textcolor{comment}{/* Accept incoming subscription
       request for this friend*/}
        linphone_friend_done(\textcolor{keyword}{friend}); \textcolor{comment}{/*commit change*/}
        linphone_core_add_friend(lc,\textcolor{keyword}{friend}); \textcolor{comment}{/* add this new friend to the buddy list*/}

\}
\textcolor{keyword}{static} \textcolor{keywordtype}{void} registration\_state\_changed(\textcolor{keyword}{struct} \_LinphoneCore *lc, 
      LinphoneProxyConfig *cfg, LinphoneRegistrationState cstate, \textcolor{keyword}{const} \textcolor{keywordtype}{char} *message)\{
                printf(\textcolor{stringliteral}{"New registration state %s for user id [%s] at proxy [%s]"}
                                ,linphone_registration_state_to_string(cstate)
                                ,linphone_proxy_config_get_identity(cfg)
                                ,linphone\_proxy\_config\_get\_addr(cfg));
\}

LinphoneCore *lc;
\textcolor{keywordtype}{int} main(\textcolor{keywordtype}{int} argc, \textcolor{keywordtype}{char} *argv[])\{
        LinphoneCoreVTable vtable=\{0\};

        \textcolor{keywordtype}{char}* dest\_friend=NULL;
        \textcolor{keywordtype}{char}* identity=NULL;
        \textcolor{keywordtype}{char}* password=NULL;

        LinphoneFriend* my\_friend=NULL;
        LinphonePresenceModel *model;

        \textcolor{comment}{/* takes   sip uri  identity from the command line arguments */}
        \textcolor{keywordflow}{if} (argc>1)\{
                dest\_friend=argv[1];
        \}
        \textcolor{comment}{/* takes   sip uri  identity from the command line arguments */}
        \textcolor{keywordflow}{if} (argc>2)\{
                identity=argv[2];
        \}
        \textcolor{comment}{/* takes   password from the command line arguments */}
        \textcolor{keywordflow}{if} (argc>3)\{
                password=argv[3];
        \}
        signal(SIGINT,stop);
\textcolor{comment}{//#define DEBUG}
\textcolor{preprocessor}{#ifdef DEBUG}
        linphone_core_enable_logs(NULL); \textcolor{comment}{/*enable liblinphone logs.*/}
\textcolor{preprocessor}{#endif}
        \textcolor{comment}{/*}
\textcolor{comment}{         Fill the LinphoneCoreVTable with application callbacks.}
\textcolor{comment}{         All are optional. Here we only use the both notify\_presence\_received and
       new\_subscription\_requested callbacks}
\textcolor{comment}{         in order to get notifications about friend status.}
\textcolor{comment}{         */}
        vtable.notify_presence_received=notify\_presence\_recv\_updated;
        vtable.new_subscription_requested=new\_subscription\_requested;
        vtable.registration_state_changed=registration\_state\_changed; \textcolor{comment}{/*just in case sip proxy is used*/}

        \textcolor{comment}{/*}
\textcolor{comment}{         Instantiate a LinphoneCore object given the LinphoneCoreVTable}
\textcolor{comment}{        */}
        lc=linphone_core_new(&vtable,NULL,NULL,NULL);
        \textcolor{comment}{/*sip proxy might be requested*/}
        \textcolor{keywordflow}{if} (identity != NULL)  \{
                \textcolor{comment}{/*create proxy config*/}
                LinphoneProxyConfig* proxy\_cfg = linphone_proxy_config_new();
                \textcolor{comment}{/*parse identity*/}
                LinphoneAddress *from = linphone_address_new(identity);
                LinphoneAuthInfo *info;
                \textcolor{keywordflow}{if} (from==NULL)\{
                        printf(\textcolor{stringliteral}{"%s not a valid sip uri, must be like sip:toto@sip.linphone.org \(\backslash\)n"},identity
      );
                        \textcolor{keywordflow}{goto} end;
                \}
                \textcolor{keywordflow}{if} (password!=NULL)\{
                        info=linphone_auth_info_new(
      linphone_address_get_username(from),NULL,password,NULL,NULL,NULL); \textcolor{comment}{/*create authentication structure from
       identity*/}
                        linphone_core_add_auth_info(lc,info); \textcolor{comment}{/*add authentication info to LinphoneCore*/}
                \}

                \textcolor{comment}{// configure proxy entries}
                linphone_proxy_config_set_identity(proxy\_cfg,identity); \textcolor{comment}{/*set identity with user name and
       domain*/}
                linphone_proxy_config_set_server_addr(proxy\_cfg,
      linphone_address_get_domain(from)); \textcolor{comment}{/* we assume domain = proxy server address*/}
                linphone_proxy_config_enable_register(proxy\_cfg,TRUE); \textcolor{comment}{/*activate registration for this
       proxy config*/}
                linphone_proxy_config_enable_publish(proxy\_cfg,TRUE); \textcolor{comment}{/* enable presence satus publication
       for this proxy*/}
                linphone_address_unref(from); \textcolor{comment}{/*release resource*/}

                linphone_core_add_proxy_config(lc,proxy\_cfg); \textcolor{comment}{/*add proxy config to linphone core*/}
                linphone_core_set_default_proxy(lc,proxy\_cfg); \textcolor{comment}{/*set to default proxy*/}


                \textcolor{comment}{/* Loop until registration status is available */}
                \textcolor{keywordflow}{do} \{
                        linphone_core_iterate(lc); \textcolor{comment}{/* first iterate initiates registration */}
                        ms\_usleep(100000);
                \}
                \textcolor{keywordflow}{while}(  running && linphone_proxy_config_get_state(proxy\_cfg) == 
      LinphoneRegistrationProgress);

        \}

        \textcolor{keywordflow}{if} (dest\_friend) \{
                my\_friend = linphone_core_create_friend_with_address(lc, dest\_friend); \textcolor{comment}{/*creates friend
       object from dest*/}
                \textcolor{keywordflow}{if} (my\_friend == NULL) \{
                        printf(\textcolor{stringliteral}{"bad destination uri for friend [%s]\(\backslash\)n"},dest\_friend);
                        \textcolor{keywordflow}{goto} end;
                \}

                linphone_friend_enable_subscribes(my\_friend,TRUE); \textcolor{comment}{/*configure this friend to emit
       SUBSCRIBE message after being added to LinphoneCore*/}
                linphone_friend_set_inc_subscribe_policy(my\_friend,
      LinphoneSPAccept); \textcolor{comment}{/* Accept incoming subscription request for this friend*/}
                linphone_core_add_friend(lc,my\_friend); \textcolor{comment}{/* add my friend to the buddy list, initiate
       SUBSCRIBE message*/}

        \}

        \textcolor{comment}{/*set my status to online*/}
        model = linphone_presence_model_new();
        linphone_presence_model_set_basic_status(model, 
      LinphonePresenceBasicStatusOpen);
        linphone_core_set_presence_model(lc, model);
        linphone_presence_model_unref(model);

        \textcolor{comment}{/* main loop for receiving notifications and doing background linphone core work: */}
        \textcolor{keywordflow}{while}(running)\{
                linphone_core_iterate(lc); \textcolor{comment}{/* first iterate initiates subscription */}
                ms\_usleep(50000);
        \}

        \textcolor{comment}{/* change my presence status to offline*/}
        model = linphone_presence_model_new();
        linphone_presence_model_set_basic_status(model, 
      LinphonePresenceBasicStatusClosed);
        linphone_core_set_presence_model(lc, model);
        linphone_presence_model_unref(model);
        linphone_core_iterate(lc); \textcolor{comment}{/* just to make sure new status is initiate message is issued */}

        linphone_friend_edit(my\_friend); \textcolor{comment}{/* start editing friend */}
        linphone_friend_enable_subscribes(my\_friend,FALSE); \textcolor{comment}{/*disable subscription for this friend*/}
        linphone_friend_done(my\_friend); \textcolor{comment}{/*commit changes triggering an UNSUBSCRIBE message*/}

        linphone_core_iterate(lc); \textcolor{comment}{/* just to make sure unsubscribe message is issued */}

end:
        printf(\textcolor{stringliteral}{"Shutting down...\(\backslash\)n"});
        linphone_core_destroy(lc);
        printf(\textcolor{stringliteral}{"Exited\(\backslash\)n"});
        \textcolor{keywordflow}{return} 0;
\}

\end{DoxyCodeInclude}
 