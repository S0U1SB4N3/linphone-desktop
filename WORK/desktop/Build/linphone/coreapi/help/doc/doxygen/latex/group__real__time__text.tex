\section{Real Time Text Receiver}
\label{group__real__time__text}\index{Real Time Text Receiver@{Real Time Text Receiver}}


This program is able to receive chat message in real time on port 5060.  


This program is able to receive chat message in real time on port 5060. 

Use realtimetext\+\_\+sender to generate chat message usage\+: ./realtimetext\+\_\+receiver


\begin{DoxyCodeInclude}

\textcolor{comment}{/*}
\textcolor{comment}{linphone}
\textcolor{comment}{Copyright (C) 2015  Belledonne Communications SARL}
\textcolor{comment}{}
\textcolor{comment}{This program is free software; you can redistribute it and/or}
\textcolor{comment}{modify it under the terms of the GNU General Public License}
\textcolor{comment}{as published by the Free Software Foundation; either version 2}
\textcolor{comment}{of the License, or (at your option) any later version.}
\textcolor{comment}{}
\textcolor{comment}{This program is distributed in the hope that it will be useful,}
\textcolor{comment}{but WITHOUT ANY WARRANTY; without even the implied warranty of}
\textcolor{comment}{MERCHANTABILITY or FITNESS FOR A PARTICULAR PURPOSE.  See the}
\textcolor{comment}{GNU General Public License for more details.}
\textcolor{comment}{}
\textcolor{comment}{You should have received a copy of the GNU General Public License}
\textcolor{comment}{along with this program; if not, write to the Free Software}
\textcolor{comment}{Foundation, Inc., 51 Franklin Street, Fifth Floor, Boston, MA  02110-1301, USA.}
\textcolor{comment}{*/}

\textcolor{preprocessor}{#include "linphone/core.h"}

\textcolor{preprocessor}{#include <signal.h>}

\textcolor{keyword}{static} bool\_t running=TRUE;

\textcolor{keyword}{static} \textcolor{keywordtype}{void} stop(\textcolor{keywordtype}{int} signum)\{
        running=FALSE;
\}



\textcolor{keywordtype}{int} main(\textcolor{keywordtype}{int} argc, \textcolor{keywordtype}{char} *argv[])\{
        LinphoneCoreVTable vtable=\{0\};
        LinphoneCore *lc;
        LinphoneCall *call=NULL;
        LinphoneChatRoom *chat\_room;
        LinphoneChatMessage *chat\_message=NULL;
        \textcolor{keyword}{const} \textcolor{keywordtype}{char} *dest=NULL;
        LCSipTransports tp;
        tp.udp\_port=LC_SIP_TRANSPORT_RANDOM;
        tp.tcp\_port=LC_SIP_TRANSPORT_RANDOM;
        tp.tls\_port=LC_SIP_TRANSPORT_RANDOM;

        \textcolor{comment}{/* take the destination sip uri from the command line arguments */}
        \textcolor{keywordflow}{if} (argc>1)\{
                dest=argv[1];
        \}

        signal(SIGINT,stop);

\textcolor{preprocessor}{#ifdef DEBUG}
        linphone_core_enable_logs(NULL); \textcolor{comment}{/*enable liblinphone logs.*/}
\textcolor{preprocessor}{#endif}

        \textcolor{comment}{/*}
\textcolor{comment}{         Instanciate a LinphoneCore object given the LinphoneCoreVTable}
\textcolor{comment}{        */}
        lc=linphone_core_new(&vtable,NULL,NULL,NULL);


        linphone_core_set_sip_transports(lc,&tp); \textcolor{comment}{/*to avoid port colliding with receiver*/}
        \textcolor{keywordflow}{if} (dest)\{
                \textcolor{comment}{/*}
\textcolor{comment}{                 Place an outgoing call with rtt enabled}
\textcolor{comment}{                */}
                LinphoneCallParams *cp = linphone_core_create_call_params(lc, NULL);
                linphone_call_params_enable_realtime_text(cp,TRUE); \textcolor{comment}{/*enable real time text*/}
                call=linphone_core_invite_with_params(lc,dest,cp);
                linphone_call_params_unref(cp);
                \textcolor{keywordflow}{if} (call==NULL)\{
                        printf(\textcolor{stringliteral}{"Could not place call to %s\(\backslash\)n"},dest);
                        \textcolor{keywordflow}{goto} end;
                \}\textcolor{keywordflow}{else} printf(\textcolor{stringliteral}{"Call to %s is in progress..."},dest);
                linphone_call_ref(call);

        \}
        \textcolor{comment}{/*wait for call to be established*/}
        \textcolor{keywordflow}{while}   (running && (linphone_call_get_state(call) == 
      LinphoneCallOutgoingProgress
                                                || linphone_call_get_state(call) == 
      LinphoneCallOutgoingInit)) \{
                linphone_core_iterate(lc);
                ms\_usleep(50000);
        \}
        \textcolor{comment}{/*check if call is established*/}
        \textcolor{keywordflow}{switch} (linphone_call_get_state(call)) \{
        \textcolor{keywordflow}{case} LinphoneCallError:
        \textcolor{keywordflow}{case} LinphoneCallReleased:
        \textcolor{keywordflow}{case} LinphoneCallEnd:
                printf(\textcolor{stringliteral}{"Could not place call to %s\(\backslash\)n"},dest);
                \textcolor{keywordflow}{goto} end;
                \textcolor{keywordflow}{break};
        \textcolor{keywordflow}{default}:
                \textcolor{keywordflow}{break};
                \textcolor{comment}{/*continue*/}
        \}

        chat\_room=linphone_call_get_chat_room(call); \textcolor{comment}{/*create a chat room associated to this call*/}

        \textcolor{comment}{/* main loop for sending message and doing background linphonecore work: */}
        \textcolor{keywordflow}{while}(running)\{
                \textcolor{keywordtype}{int} character;
                \textcolor{comment}{/*to disable terminal buffering*/}
                \textcolor{keywordflow}{if} (system (\textcolor{stringliteral}{"/bin/stty raw"}) == -1)\{
                        ms\_error(\textcolor{stringliteral}{"/bin/stty error"});
                \}
                character = getchar();
                \textcolor{keywordflow}{if} (system(\textcolor{stringliteral}{"/bin/stty cooked"}) == -1)\{
                        ms\_error(\textcolor{stringliteral}{"/bin/stty error"});
                \}
                \textcolor{keywordflow}{if} (character==0x03) \{\textcolor{comment}{/*CTRL C*/}
                        running=0;
                        \textcolor{keywordflow}{break};
                \}
                \textcolor{keywordflow}{if} (character != EOF) \{
                        \textcolor{comment}{/* user has typed something*/}
                        \textcolor{keywordflow}{if} (chat\_message == NULL) \{
                                \textcolor{comment}{/*create a new message*/}
                                chat\_message = linphone_chat_room_create_message(chat\_room,\textcolor{stringliteral}{""}); \textcolor{comment}{/*create an
       empty message*/}
                        \}
                        \textcolor{keywordflow}{if} (character == \textcolor{charliteral}{'\(\backslash\)r'}) \{
                                \textcolor{comment}{/*new line, committing message*/}
                                linphone_chat_room_send_chat_message(chat\_room,chat\_message);
                                chat\_message = NULL; \textcolor{comment}{/*reset message*/}
                        \} \textcolor{keywordflow}{else} \{
                                linphone_chat_message_put_char(chat\_message,character); \textcolor{comment}{/*send char in
       realtime*/}
                        \}
                \}
                linphone_core_iterate(lc);
                ms\_usleep(50000);
        \}
        \textcolor{keywordflow}{if} (call && linphone_call_get_state(call)!=LinphoneCallEnd)\{
                \textcolor{comment}{/* terminate the call */}
                printf(\textcolor{stringliteral}{"Terminating the call...\(\backslash\)n"});
                linphone_core_terminate_call(lc,call);
                \textcolor{comment}{/*at this stage we don't need the call object */}
                linphone_call_unref(call);
        \}

end:
        printf(\textcolor{stringliteral}{"Shutting down...\(\backslash\)n"});
        linphone_core_destroy(lc);
        printf(\textcolor{stringliteral}{"Exited\(\backslash\)n"});
        \textcolor{keywordflow}{return} 0;
\}

\end{DoxyCodeInclude}
 